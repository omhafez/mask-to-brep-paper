\section{Contribution}
\label{sec:Contribution}

The workflow we present takes as input a binary segmented image and produces a facetized boundary representation.  The segmented image is presumed to contain miscolored voxels, such that the desired b-rep will generally be topologically distinct from the input image's complex of binary-interface surfels.  The first step in our algorithm is a novel boundary point/normal estimation procedure that operates on partially-overlapping subsets of voxels, or {\em windows}.  This step provides a means for smoothing both geometric and topological irregularities, and is a distinguishing feature of our algorithm.  It is the main mechanism for robustly recovering smooth b-reps from noisy voxel colorings, \textcolor{blue}{and is the main contribution of the paper}.  The window size is selectable; it sets the length scale associated with the smoothing.  The oriented point cloud that emerges from this first step is then used in a Voronoi-based surface reconstruction procedure.  \textcolor{blue}{We chose a Voronoi-based method largely because of its guarantee of watertight results, although screened Poisson would be a reliable and effective alternative for many applications.  In this connection, a point in favor of Voronoi is its good prospects for capturing sharp features.  We provide some evidence of this in subsection 5.5.}  

It bears emphasis that, prior to the surface reconstruction step, there is no surface as such, and therefore nothing to sample.  \textcolor{blue}{In this respect, our point/normal estimation procedure is not directly comparable to methods that assume the existence of a surface, such as those based on linear regression or moving least squares.}  In effect, we refrain from assuming that the surfel complex at the interface between mono-colored voxel regions is homeomorphic to the desired b-rep.  The term ``surface synthesis'' might therefore be more appropriate than the standard ``surface reconstruction,'' which we nevertheless retain. The resulting surface is guaranteed to be manifold, whereas a guarantee of closedness can be achieved with a trivial modification.  However, depending on the details of the Voronoi surface reconstruction procedure and the desired mesh density, a subsequent smoothing and decimation step is often warranted.  For this, we use existing freely-available algorithms.

The performance of the proposed method is assessed by comparing its results to those of three comparison workflows.  The comparison workflows and the proposed one are illustrated in flowchart form in \figref{flowchart}.  We refer to each of the three with the name of a key element of the respective workflow.  They are:
\begin{enumerate}
\item  {\em Marching cubes:}  The marching cubes algorithm is applied to the image mask, followed by decimation and smoothing that is identical to what is used in our workflow.
\item  {\em Simpleware:}  The commercial software package {\em Simpleware} is used to generate a triangulated b-rep from an input binary image mask.  The software implements the enhanced volumetric marching cubes algorithm \cite{young_2008}, followed by proprietary decimation and smoothing operations.  Optimal results were obtained by selecting the ``binarise before smoothing'' option, and using 100 iterations of ``smart mask smoothing.''  Otherwise, options are set to standard or default values.
\item  {\em Screened Poisson:}  \textcolor{blue}{This workflow uses our proposed method to generate an oriented point cloud, but then replaces our Voronoi-based surface reconstruction with screened Poisson.  The screened-Poisson surface reconstruction technique \cite{kazhdan_2013} as implemented in the \textit{Meshlab} software package is used, with the software's default parameters.}  Only decimation is performed on the reconstructed surface, as the screened Poisson algorithm already performs smoothing on the indicator function.  \textcolor{blue}{By employing a known, highly effective surface reconstruction ``back end,'' this comparison workflow is intended to illustrate the effectiveness of the proposed point/normal estimation method.}
\end{enumerate}
Performance is measured quantitatively for canonical image masks defined by spheres of various radii and locations, and qualitatively using a variety of real MRI and CT scans.  For the quantitative measures, we devise separate nondimensional shape and volume errors for a sphere.  
