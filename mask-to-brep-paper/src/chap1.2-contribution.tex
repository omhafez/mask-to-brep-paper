\section{Contribution}

The workflow we present takes as input a binary segmented image and produces a facetized boundary representation.  The segmented image is presumed to contain mis-colored voxels, such that the desired b-rep will generally be topologically distinct from the input image's complex of binary-interface surfels.  The first step in our algorithm is a novel boundary point/normal estimation procedure that operates on partially-overlapping subsets of voxels, or {\em windows}.  This step provides a means for smoothing both geometric and topological irregularities, and is a distinguishing feature of our algorithm.  It is the main mechanism for robustly recovering smooth b-reps from noisy voxel colorings.  The window size is selectable; it sets the length scale associated with the smoothing.  The oriented point cloud that emerges from this first step is then used in a Voronoi-based surface-reconstruction procedure.  It bears emphasis that, prior to this surface-reconstruction step, there is no surface as such, and therefore nothing to sample.  In effect, we refrain from assuming that the surfel complex at the interface between mono-colored voxel regions is homeomorphic to the desired b-rep.  The term ``surface synthesis'' might therefore be more appropriate than the standard ``surface reconstruction,'' which we nevertheless retain. The resulting surface is guaranteed to be closed and manifold.  However, depending on the details of the Voronoi surface-reconstruction procedure and the desired mesh density, a subsequent smoothing and decimation step is often warranted.  For this, we use existing freely-available algorithms.

In section XX, we compare the performance of

The performance of the proposed method is assessed by comparing its results to those of different approaches, namely: 1) marching cubes, 2) Simpleware, a state-of-the-art commercial software, and 3) screened Poisson surface reconstruction. For the marching cubes approach, the same decimation and smoothing steps are performed as in the proposed method for fair comparison. For Simpleware, optimal results were obtained from {Simpleware} by selecting the ``binarise before smoothing" option and performing 100 iterations of ``smart mask smoothing''; standard options were chosen otherwise. For the screened Poisson surface reconstruction approach, oriented point clouds generated as part of the proposed method are provided as input to the surface reconstruction method. The implementation of screened Poisson surface reconstruction in \textit{Meshlab} was utilized with its default parameters.  Performance is measured quantitatively for canonical image masks defined by spheres of various radii and locations, and qualitatively using a variety of real MRI and CT scans.

Provided the context of the image-based modeling workflow, we present our Voronoi-based surface generation algorithm in the next section. A performance assessment follows, which includes comparison to other approaches.  We consider both quantitative error measures as well as qualitative, visual comparison of b-rep results.  The paper is closed with a summary and discussion of future directions for this work.