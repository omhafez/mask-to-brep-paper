\section{Closure}

The proposed method offers a novel approach to robustly generate watertight surface meshes from binary image masks for purposes of physics-based simulation, 3D printing, or visualization. The method densely samples the image and locally approximates the interface between neighboring materials based on a template surface. The optimal orientation of the template surface is obtained via minimization of a weighted error function based on difference in the zeroth, first, and second moments of volume between the voxelated sampling window and the region created by the approximating template. Oriented points are placed based on each local surface optimization. By combining the results from each sampling window, the problem becomes the more tractable task of surface reconstruction from a dense oriented point cloud. In essence, the sampling process transforms the input from a fairly coarse data description the form of an image mask limited by the voxel resolution, to a much richer dataset in the form an oriented point cloud that provides multiples more of data points describing the interface of interest. \\ \\
%
The proposed method utilizes Voronoi partitioning as a means to perform surface reconstruction from the point cloud. Namely, Voronoi sites are placed on either side of each point along its orientation, and material identifiers are assigned based on which side of the interface they lie. A Voronoi partition is performed and the interface is extracted based on facets who share Voronoi sites of differing material ID. This technique takes advantage of the robust qualities of a Voronoi partition, namely, that the resulting surface is guaranteed to be watertight. Additional smoothing and decimation are performed as is typical in this field due to the noisy nature of the input data. Comparisons to commerical software show that the proposed approach performs competitively in honoring shape and volume for a suite of canonical examples, and robustly and visually for a variety of examples from real MRI and CT scans. \\ \\
%
Besides the benefit of robustness, Voronoi parititioning also opens the door nicely to handle sharp edges and corners, as well as multi-material interfaces. With the current framework established, the method now has the ability to extend to those use cases by introducing new interface templates to the optimization problem. Specifically, an edge template would allow for the introduction of three Voronoi sites, and a corner template would allow for the introduction of four Voronoi sites. Doing so holds the promise of generating b-reps that well approximate the surface of objects even if they have sharp edges and corners. This would particularly hold value for cases where non-destructive analysis on as-built engineering parts or legacy parts is desired. Similarly, a template with high curvature that potentially introduces mutliple Voronoi site pairs at one time could dramatically improve the method's performance in regions of high curvature. Lastly, multi-material interface templates could naturally be added as well, where the Voronoi site is easily assigned to whichever material it belongs. Attractively, Voronoi paritioning would no more difficult in that scenario. Indeed, the introduction of these new templates would extend the method well beyond the current paradigm of effectively transforming the problem into reconstructing a surface from a point cloud, but rather generating a much richer data set for each sampling window than just a single oriented point.