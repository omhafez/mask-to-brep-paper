\section{Closure}

We propose a workflow for creation of a watertight, manifold, facetized boundary representation given a binary image mask containing noise.  Our target application is physics-based simulation, for which a volume mesh must of course be generated by separate means (e.g. tetrahedral advancing-front generator).  The method densely samples the image mask and locally approximates the interface between neighboring materials based on a local template surface. The optimal orientation of the template surface is obtained via minimization of a weighted error function. The error function is based on the difference in the zeroth, first, and second polynomial moments of the voxelated sampling window and the region created by the approximating template. Oriented points are then placed with the help of each optimal local surface. In aggregate, the local templates produce an oriented point cloud whose density is controlled by the selected window-size and window-overlap parameters.  This procedure provides for effective smoothing of the surface normal while retaining the surface ``position'' data inherent in the image mask.  The oriented-point-cloud procedure does not assume the existence of a target surface with the desired topology.  Instead, it in effect synthesizes a surface that ignores fine-scale islands and holes in the image mask, as often occur in noisy segmented images.  This is an important distinguishing feature of our method.

With the oriented point cloud in hand, the method utilizes Voronoi partitioning as a means of performing surface reconstruction from the point cloud.  Specifically, Voronoi sites are placed on either side of each point along its normal vector, and material identifiers (or colors) are assigned to the sites. A Voronoi partition of 3D space is performed, and the b-rep is extracted in the form of the complex of facets that share Voronoi sites of differing material ID. This technique takes advantage of some useful properties of the Voronoi tessellation, including being guaranteed manifold. Following construction of the Voronoi surface, decimation and smoothing are performed.  Comparisons to three other workflows show that the proposed approach performs competitively with respect to shape and volume of analytic spherical surfaces. The approach was also seen to perform well for a variety of examples from real MRI and CT scans.

With the current framework established in its basic form, work has turned to extending the method to geometries with sharp corners or edges, high-curvature regions, and multiple materials.  The extension to sharp features is accomplished through introduction of new local interface templates to supplement the simple planar one used herein.  Specifically, an edge template would allow for the introduction of three Voronoi sites, and a corner template would produce four. These enhancements hold the promise of generating b-reps that naturally capture surfaces with sharp edges and corners, even when coarse windowing is selected in the point-cloud-generation step.  It is largely with these properties in mind that we chose a Voronoi-based approach for the surface reconstruction step, rather than e.g. a Delaunay-based one.   More generally, templates can be designed with multiple oriented points, such that high local curvature can be obtained where dictated by the image mask.  

A limitation of our method at its present stage of development is that it is limited to binary images.  While the Voronoi-based surface reconstruction technique is readily adaptable to the multi-material case, the window-based point-cloud step would require more work.  Specifically, new and probably complex local interface templates would have to be introduced.  While there are no theoretical constraints that would prevent this, the practicalities have yet to be explored.