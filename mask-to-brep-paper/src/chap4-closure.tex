\section{Closure}

The proposed method offers a novel way to robustly generate watertight surface meshes from binary image masks for the purpose of physics-based simulation, 3D printing, or visualization. For purposes of physics-based simulation, the surface mesh may be used as input to any number of tetrahedral mesh generators to produce a volume mesh. The method densely samples the image and locally approximates the interface between neighboring materials based on a local template surface. The optimal orientation of the template surface is obtained via minimization of a weighted error function. The error function is based on the difference in the zeroth, first, and second polynomial moments of the voxelated sampling window and the region created by the approximating template. Oriented points are then placed with the help of each optimal local surface. In aggregate, the local templates produce an oriented point cloud whose density is indexed to the voxel resolution. This procedure provides for effective smoothing of the surface normal while retaining the surface ``position'' data inherent in the image mask. \\ \\
%
With the oriented point cloud in hand, the method utilizes Voronoi partitioning as a means of performing surface reconstruction from the point cloud. Namely, Voronoi sites are placed on either side of each point along its orientation vector, and material identifiers are assigned based on which side of the boundary they lie. A Voronoi partition of 3D space is performed, and the boundary is extracted in the form of facets that share Voronoi sites of differing material ID. This technique takes advantage of the robust qualities of Voronoi partitioning, including guaranteed watertightness. Following construction of the Voronoi surface, decimation and smoothing are performed, as is commonplace due to the noisy nature of the input data. Comparisons to \textcolor{purple}{other methods} show that the proposed approach performs competitively with respect to shape and volume of analytic spherical surfaces. The approach performs robustly for a variety of examples from real MRI and CT scans, with results that visually compare well to those of the \textcolor{purple}{different approaches}. \\ \\
%
With the current framework established in its basic form, work has turned to extending the method to geometries with sharp corners or edges, high-curvature regions, and multi-material interfaces. The extension to sharp features is accomplished through introduction of new local interface templates. Specifically, an edge template would allow for the introduction of three Voronoi sites, and a corner template would produce four. These enhancements hold the promise of generating b-reps that well-approximate the surfaces of objects with sharp edges and corners. This may be of value in the non-destructive analysis of as-built engineering parts or legacy parts. More generally, templates can be designed with multiple oriented points, such that high local curvature can be obtained where dictated by the image mask.  Lastly, multi-material templates could be included in a comprehensive implementation, where the local optimization procedure would include assignment of a material index to each Voronoi site. The overall structure of the algorithm remains the same under these enhancements; for example, Voronoi partitioning would be identical to the base case. The introduction of these new templates would provide richer data from each sampling window, ultimately resulting in higher-quality surface meshes for a wider range of inputs. \\ \\
%
Fit this in above: \\ \\
%
Regions of high curvature, or even sharp corners and edges, can be addressed via an enhancement of the point-cloud-generation algorithm described previously. The enhancement involves including additional templates when locally approximating boundaries, such that more than two Voronoi sites are generated for each sampling window. For example, three Voronoi sites can be used to produce an edge, and four sites can be used to produce a corner. In this manner, multiple boundary point/normal pairs for a single sampling window would naturally occur to reconstruct regions of high curvature.
\color{purple} Limitations
\cite{edelsbrunner1994}, \cite{coeurjolly2021}, \cite{fleishman2005}, \cite{brochu2010}, \cite{berge2019}, \cite{dey2003}, \cite{amenta2000}, \cite{dey2011}, \cite{boltcheva2017}, \cite{boltcheva2009}, \cite{kazhdan_2008}, \cite{kazhdan_2013}, \cite{levin2004}

\color{black}