\section{Incremental Kinematics}
%

\newcommand{\ds}{\displaystyle}
\newcommand{\mparen}[1]{\displaystyle \big( #1 \big)}
\newcommand{\mbrack}[1]{\displaystyle \big[ #1 \big]}
\newcommand{\bparen}[1]{\displaystyle \bigg( #1 \bigg)}
\newcommand{\bbrack}[1]{\displaystyle \bigg[ #1 \bigg]}

The Jaumann rate of stress for a linear hypoelastic material is characterized by:

$$ \overset{\Delta}{\sigma} \equiv 
\dot{\sigma} + \sigma \cdot W - W \cdot \sigma = 
C : D $$

where $ D = \frac{1}{2} (L + L^\top), \; W = \frac{1}{2} (L - L^\top)$, and $L = \dot{F}
F^{-1} $. We would like to know how the stress state changes after applying
a prescribed deformation. Our approach is as follows: parameterize the deformation
into separate stages of pure stretch and pure rotation. The advantage of this
approach is that it allows the terms in the stress rate above to be integrated
separately

\begin{align*}
\dot{\sigma} = 
\begin{cases}
C : D & \qquad \text{W vanishes for pure stretch} \\
W \cdot \sigma - \sigma \cdot W & \qquad \text{D vanishes for pure rotation}
\end{cases}
\end{align*}

Simple analytic solutions exist for each of these ordinary differential equations, 
provided that $D, W$ are constant ($C$ is also assumed to be constant):

\begin{align*}
\dot{\sigma}(t) = C : D, \; \sigma(0) = \sigma_i 
\qquad 
& \Longrightarrow 
\qquad 
\sigma_{i+\frac{1}{2}}  = \sigma_i + C : D \Delta t \\ \\
\dot{\sigma}(t) = W \cdot \sigma - \sigma \cdot W, \; \sigma(0) = \sigma_{i+\frac{1}{2}} 
\qquad 
& \Longrightarrow 
\qquad 
\sigma_{i+1} = \exp(W \Delta t) \; \sigma_{i+\frac{1}{2}} \; \exp(W \Delta t)^{\top}
\end{align*}

Combining these two results gives us our stress update procedure:

$$ \sigma_{i+1} = \exp(W \Delta t) \; \bparen{\sigma_i + C : D \Delta t} \; \exp(W \Delta t)^{\top} $$

All that remains is to find reasonable approximations for $D, W$. To that end, we
consider the polar decomposition of the incremental deformation gradient, 
$F = R \; U$. After taking a time derivative of $F$, and substituting into the
definition of $L$, we get:

$$ L \equiv D + W = \dot{R} \; R^\top + R \; \dot{U} \; U^{-1} \; R^\top $$

When considering the separate stages of pure stretch and pure rotation, this
equation reduces to:

\begin{align*}
\begin{cases}
D = \dot{U} \; U^{-1} & \qquad \text{pure stretch} \\
W = \dot{R} \; R^\top & \qquad \text{pure rotation}
\end{cases}
\end{align*}

These are constant-coefficient, ordinary differential equations
than can be used to determine $D$ and $W$. The
general solution for equations of this form is

$$ 
Y = \dot{X} X^{-1}, \; X(0) = \mathbf{1} 
\qquad 
\Longrightarrow 
\qquad 
X(\Delta t) = \exp(Y \Delta t)
$$

Which, when applied to the pure stretch and pure rotation versions of the problem
gives us our definitions of $D, W$:

$$ D = \frac{1}{\Delta t} \log(U), \qquad W = \frac{1}{\Delta t} \log(R) $$

Substituting these back into our stress update procedure gives 

$$\boxed{\sigma_{i+1} = R \; \bparen{\sigma_i + C : \log(U)} \; R^\top}$$




%%%%%%%%%%%%%%%%%%%%%%%%%%%%%%%%%%%%%%%%%%%%%%%
%%%%%%%%%%%%%%%%%%%%%%%%%%%%%%%%%%%%%%%%%%%%%%%
\subsection{Description}
\label{Description}

%%%%%%%%%%%%%%%%%%%%%%%%%%%%%%%%%%%%%%%%%%%%%%%
%%%%%%%%%%%%%%%%%%%%%%%%%%%%%%%%%%%%%%%%%%%%%%%
\subsection{Implementation Approaches}
\label{Implementation Approaches}
%%%%%%%%%%%%%%%%%%%%%%%%%%%%%%%%%%%%%%%%%%%%%%%
%%%%%%%%%%%%%%%%%%%%%%%%%%%%%%%%%%%%%%%%%%%%%%%
\subsubsection{Hughes-Winget}
\label{Hughes-Winget}

\subsubsection{Rashid}
\label{Rashid}

hi