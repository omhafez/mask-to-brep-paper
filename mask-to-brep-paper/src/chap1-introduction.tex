\section{Introduction}

1. read simpleware paper intro for inspo \\
%
2. go through each chapter in disseration and grab summaries/lit reviews from there \\ \\
%
Image-based modeling and simulation is becoming an increasingly important analytic and predictive tool for patient-specific medical applications, large-scale \textit{in silico} trials, optimized medical device design, custom surgical guides and implants, improving diagnosis and treatment, computer-assisted surgery, surgeon training, and as-built analysis of existing parts. Whereas traditional physics-based modeling and simulation typically uses NURBS-based surfaces generated in CAD software as a starting point to define geometries of the objects of interest, the image-based analog defines geometries from imaging data, typically in the form MRI or CT. The topic covers a broad spectrum of fields, including image processing, computational geometry, numerical methods, and mechanics. The workflow entails: image acquisition, image segmentation, image-based mesh generation, and finally additive manufacturing and/or physics-based modeling and simulation. This paper focuses on a component of image-based mesh generation, namely surface generation. The other steps are briefly mentioned here in the introduction, but otherwise we will focus on image-based meshing and specifically surface generation from a segmented image. \\ \\
%
Medical imaging is a broad and deep field that we won't focus on here. For our purposes, it provides a three-dimensional rectilinear grid of point intensity values. In the case of MRI, that value is a weighted proton density, which effectively measures water-content. In the case of CT, that value is the attenuation of an x-ray beam source, which effectively measures density. For typical applications, we can expect O(mm) resolution or in some cases even better. \\ \\
%
Image segmentation is the process of partitioning an image into non-overlapping regions corresponding to different tissues or objects in an image. Contrast differences between neighboring tissues can be difficult to identify due in part to noise and artifacts, and thus image processing in the form of smoothing, low-pass filters, or resampling is performed to improve the effectiveness of the image segmentation technique. A few popular techniques in image segmentation include thresholidng, deformable models, atlas-guided approaches, and often for practical purposes, manual fine tuning. (sources for each, and a source for a lit review).\\ \\
%
Image-based mesh generation, the focus of this paper, involves generating geometrical meshes suitable for simulation based on segmented images. Image-based meshing is often separated into two steps: 1) surface generation,  and 2) conventional CAD-based volume meshing. A novel Voronoi-based surface generation technique is presented, and a brief review of CAD-based meshing techniques is covered. A novel image-based meshing tool \textit{Shabaka} is introduced, that makes important improvements to the speed and robustness of generating meshes from image data. Lit review, brief summary of what we do.\\ \\
%
Finally, applications -> rapid prototyping such as for surgery planning, more on patient specific modelign and simulation, potentially list applications in nearly every body part in the human body. Repeat analysis of as-built parts? \\ \\
%
The proposed method is described, followed by a performance assessment in comparison to a state-of-the-art commercial software, both for a variety of canonical sphere cases, as well as for image masks from real MRI and CT data, and the paper is closed out with a discussion of future directions for this work.
