\section{Introduction}

1. go through each chapter in disseration and grab summaries/lit reviews from there \\ \\
%
Image-based modeling and simulation is becoming an increasingly important analytic and predictive tool for variety of medical and engineering applications. Just some examples include: patient-specific diagnosis and treatment, large-scale \textit{in silico} trials, medical device design, computer-assisted surgery, and as-built analysis of existing parts. Whereas traditional physics-based modeling and simulation typically uses NURBS-based surfaces generated in CAD software as a starting point to define geometries of the objects of interest, the image-based analog defines geometries from imaging data, typically in the form MRI or CT. The field of image-based modeling and simulation covers a broad spectrum of topics, including image processing, computational geometry, numerical methods, and continuum mechanics. The workflow entails: image acquisition, image segmentation, image-based mesh generation, and finally additive manufacturing and/or physics-based modeling and simulation. This paper focuses on one component of image-based mesh generation, namely: surface generation. The other steps are briefly mentioned here in the introduction, but otherwise the focus shall remain on surface generation from a segmented image. [NEED TO DEFINE B-REP HERE?] \\ \\
%
Medical imaging is the process of generating discrete image representations of the object(s) of interst. For our purposes, the acquired image provides as input a three-dimensional rectilinear grid of point intensity values in the field of view. In the case of MRI, that value is a weighted proton density, which effectively measures water-content. In the case of CT, that value is the attenuation of an x-ray beam source, which effectively measures density. For typical applications, we can expect O(mm) resolution or in some cases even better. \\ \\
%
Image segmentation is the process of partitioning an image into non-overlapping regions corresponding to different tissues or objects in an image. Contrast differences between neighboring tissues can be difficult to identify due in part to noise and artifacts, and thus image processing in the form of smoothing, filtering, or resampling is performed to improve the effectiveness of the image segmentation technique. A few popular techniques in image segmentation include simple thresholding, level set methods~\cite{malladi_1995, sethian_1996}, machine learning-based approaches~\cite{litjens_2017}, and often in practical cases, manual fine tuning.\\ \\
%
Image-based mesh generation, the focus of this paper, involves generating geometrical meshes suitable for simulation based on segmented images. Image-based meshing is often separated into two steps: 1) surface generation,  and 2) conventional CAD-based volume meshing. A novel Voronoi-based surface generation technique is presented, and a brief review of CAD-based meshing techniques is covered. A novel image-based meshing tool \textit{Shabaka} is introduced, that makes important improvements to the speed and robustness of generating meshes from image data. Lit review, brief summary of what we do. Don't mention Shabaka. Need to briefly mention 3D meshing. \\ \\
%
Finally, applications for this work range from 3D printing to visualization to finite element analysis (FEA). In the case of 3D printing and visualization, b-reps can be used for surgery planning, custom surgical guides and implants, rapid prototyping, and generation of CAD surfaces from exsiting parts. Meanwhile, image-based simulation has been performed on nearly every major organ in the body, and ultimately shows great promise in advancing both personalized medicine as well as complementing large-scale clinical trials. In addition, image-based simulation offers the ability to perform non-destructive analysis on as-built engineering parts. [MORE SOURCES]\\ \\
%
The proposed method is described herein, followed by a performance assessment in comparison to a state-of-the-art commercial software. Comparisons are made for a variety of canonical cases as well as for cases from real MRI and CT data. The paper is closed with a summary and discussion of future directions for this work.
