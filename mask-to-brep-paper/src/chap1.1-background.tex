\section{Background and Related Work}

The input to the workflow we propose is a segmented image, or ``image mask.''
Image segmentation is the process of partitioning a 3D cuboidal array of cells, or voxels, into non-overlapping regions corresponding to different tissues or materials.  The raw CT or MRI image consists of intensity values for each voxel, in which noise and artifacts are generally present.  The image-segmentation process seeks to assign two or more discrete ``colors'' to the voxels based on the intensity values.  Among the main segmentation techniques are thresholding, level set methods~\cite{malladi_1995, sethian_1996}, and machine-learning-based approaches~\cite{litjens_2017}.  Our present focus is on binary image masks wherein two colors are envisaged, one indicating material and the other void.  In medical applications as well as some others, smoothing, filtering, and resampling under expert manual control are usually required to achieve the best results.  

This article proposes a process for creating a facetized b-rep from a noisy image mask, as a precursor to the generation of a volumetric mesh (e.g. finite elements) to support physics-based simulation.  The b-rep's facets must generally register with the finite elements (or other) of the mesh.  Automatic unstructured tet-meshing beginning with a triangulated b-rep can be accomplished via advancing-front-type methods \cite{jin_1993, lohner_1988} or by Delaunay tetrahedralization \cite{lohner_1997}. Updegrove \textit{et al.}~\cite{updegrove_2016} used a \textit{lofting} technique on a series of 2D segmentations to generate models of blood vessels. In lofting, an approximate centerline is constructed, and 2D segmentations are combined using spline interpolation to generate surfaces. Unstructured tetrahedral meshes are then generated from the surfaces. Lofting is known as a 2.5D approach because stacked contours of neighboring slices cannot capture arbitrary 3D geometry. 
 
Young \textit{et al.}~\cite{young_2008} used a \textit{direct meshing} approach to generate meshes from multi-material image masks, without the intermediate step of creating a surface mesh. They combined the surface generation and mesh generation stages into one process via their \textit{enhanced volumetric marching cubes} approach.  The authors extended their volumetric algorithm to handle intersections of up to eight regions in one grid cell -- the maximum number of intersections that can occur for a Cartesian grid.  The resulting mesh is \textit{mixed hex-tet}, with hexahedra in the form of the original voxels on the interior, tetrahedra near boundaries, and pyramids where needed to transition from hexes to tets. The authors also utilized partial volume-based interpolation to improve surface smoothness. Additional efforts to reduce the mesh size were made by performing an octree-based approach to collect neighboring interior hexahedral elements into larger elements.  The result is nevertheless a surface mesh that is very fine, which constrains the size of interior elements in the volume mesh.  Indeed, the default and recommended procedure in their software \textit{Simpleware} (known as \textit{+FE Free}) is to follow the two-step paradigm of b-rep generation followed by CAD-based volumetric meshing as described previously. Specifically, the software extracts the surfaces from the extended volumetric marching cubes step, performs multi-part decimation and smoothing~\cite{egst}, and finally uses a conventional CAD-based tetrahedral mesher to generate fully tetrahedral meshes from the facetized b-rep.

More broadly, many image-based volumetric meshing approaches rely on the \textit{marching cubes} algorithm~\cite{lorensen_1987} and its variants for purposes of the image-mask-to-b-rep step.  (The first two of the three comparison techniques described in Figure \ref{} incorporate marching-cubes variants.)  Marching cubes assumes the existence of an implicit isosurface, which can be defined on the voxel grid taken as a structured hex mesh.  Triangular patches are generated to approximate the intersection of this isosurface with each grid cell; the triangles are then connected together to form a continuous surface. B-reps generated via marching cubes generally exhibit fairly strong aliasing, thus smoothing is often applied to the surface mesh.  Ideally, the smoothing technique attempts to preserve the volume, as with HC Laplacian smoothing\cite{vollmer_1999} and Taubin smoothing \cite{taubin1995signal, taubin_1995}.  As first proposed, the MC algorithm was applicable only to binary image masks, but has since been extended to multi-material images.  See \cite{} for a review and further references.

Variations and enhancements of MC have been proposed that seek to preserve sharp features while retaining other favorable properties.  Labsik \textit{et al.} \cite{labsik_2002} extracted a coarse surface using marching cubes, followed by iterative remeshing to finer resolutions where features are detected.  Kobbelt \textit{et al.} \cite{kobbelt_2001} modified standard MC in two ways:  enhancing the discrete distance field that defines the underlying isosurface, and adding supplemental sampling points to detect edges and corners. Depending on the application, though, surface smoothness, and in particular diminished aliasing, may well be more important than sharp-feature preservation. Lempitsky \cite{lempitsky_2010}, for example, enforced higher-order smoothness by modifying the isosurface function and solving a convex quadratic optimization problem for each grid cell.

Meyer \textit{et al.}~\cite{meyer_2008} employed a \textit{particle-based sampling} technique for multi-material volumes. Surface samples (called \textit{particles}) are constrained to the zero set of an implicit function. The distances between particles are locally adapted to create higher densities of points near surface features. A Delaunay tetrahedralization of the sampling points is computed, and each tetrahedron is assigned a material label. The surface mesh is generated by extracting the faces bounded by tetrahedra with different material labels. Finally, a conventional CAD-based tetrahedral mesher is employed to produce an analysis-ready mesh. The approach faithfully and robustly captures the geometries of complex material interfaces. However, the authors report impractically long run times.
Examples of other image-based meshing techniques include \cite{fang_2009, mohamed_2004, jermyn_2013, boissonnat_2009}.

In the matter of explicit b-rep creation from noisy discrete data, two key features can be cited:  one concerns the type or nature of the discrete input; and the other, the extent of noise in the input that can be tolerated, and its affects on the output b-rep.  Our workflow takes as input a segmented image that generally contains mis-colored voxels.  As such, the topology of the ``literal'' image is generally distinct from that of the desired, smoothed 3D domain.  This is a somewhat subtle but important departure from the classical problem of surface reconstruction, which generally assumes the {\em existence} of a surface that is sampled -- though perhaps with noise -- by the input points.  The classical problem has been extensively treated.  In an important early work (Amenta 2000), a point set strictly on an assumed closed and manifold surface forms the input, while the output is a Delaunay triangulation of the input points.  This work is notable for the theoretical guarantees it presents, although there are requirements on the point density that are often difficult to achieve in practice, particularly near sharp features.  In (Boltcheva 2017), the input is again a clean point cloud, but the surface is reconstructed by restricting the Voronoi diagram of the points to a collection of disks centered on the input points instead of by Delaunay triangulation.

Among the most widely-used surface reconstruction algorithms for binary images are the \textit{Power Crust} and \textit{Poisson surface reconstruction} techniques. Amenta \textit{et al.} \cite{amenta_2001} presented the Power Crust algorithm for reconstructing watertight surfaces from unoriented point clouds. The algorithm first constructs the \textit{medial axis transform} from the point cloud, which consists of the set of maximal balls completely contained in the interior of the surface. The algorithm then applies an inverse transform to approximate the medial axis and produce a piecewise linear surface. The algorithm does not perform well when the point set is not sufficiently dense. Kazhdan \textit{et al.} presented \textit{Poisson surface reconstruction}~\cite{kazhdan_2008} and subsequently \textit{screened Poisson surface reconstruction}~\cite{kazhdan_2013} for oriented point clouds. They compute an indicator function defined as 1 for points inside the surface and 0 for points outside, whose gradient approximates the vector field defined by the oriented point cloud. This amounts to solving a Poisson problem for an implicit indicator function, followed by application of marching cubes to polygonize the surface. The \textit{screened} approach provides a soft constraint that encourages the reconstructed isosurface to pass through the input points. In practice, the screened approach is more robust than the original Poisson approach in generating manifold surfaces from complex point clouds. The technique produces smooth surfaces that exhibit resiliency to noise, outliers, and undersampling that makes it perhaps the best available method in many cases. \\ \\


Other point-cloud-based techniques seek to accommodate a level of noise in the point coordinates, instead of strictly interpolating assumed on-surface points.  In (Fleisman 2005), moving-least-squares approximation is used to construct projections of the input points onto smooth patches, where the points for any given patch are selected using an innovative statistical method.  The MLS-based projection is similar to that proposed by (Levin 2004), but is applied patchwise instead of globally.  This raises the possibility of capturing sharp features. Dey (2003) proposed a ``tight cocone'' method for fixing surfaces that have holes due to localized insufficient point density.  This algorithm produces an interpolating surface via Delaunay triangulation without introducing additional points.  In a further development (Dey 2011), a variant of the cocone algorithm is applied to collections of local patches to reduce memory requirements.  

Coeurjolly (2021) considers multi-colored segmented-image inputs instead of point clouds.  Their algorithm seeks to recover a smooth b-rep from the stepped one composed of the input image mask's surfels.  This is done by moving surfel vertices under the control of a quadratic minimum problem.  The quadratic function measures the departure of the final facet normals from separately-computed smooth ideal values.  The result is topologically identical to the stepped b-rep; this algorithm is therefore not ideal when image pollution in the form of arbitrary mis-colored voxels cannot be ruled out.  But in the absence of such pollution, the results are visually pleasing.  Boltcheva (2009) also consider multi-colored image masks, with particular emphasis on preservation of zero- and one-dimensional junctions where multiple colors meet.  Their algorithm extracts these junctions from the segmented image and then invokes Delaunay refinement (Pons et al. 2007) to construct a b-rep or a tetrahedral volume mesh that strictly preserves the junctions.  As with Coeurjolly (2021), the approach is suitable when the topology of the segmented image is clean, and should therefore be preserved in the output b-rep.  

Berge (2019) provides an example of Voronoi-based surface and volume meshing given data that defines free surfaces and material interfaces.  In this work, an oil reservoir defined by intersecting planes and cylindrical features constitutes the starting point.  These geometric features must be known analytically.  The objective of their algorithm is to select Voronoi sites so that point- and line-intersections, as well as surface patches, that are present in the input are reproduced as Voronoi vertices, edges, and facets, respectively.  As a mechanism for surface reconstruction, the Voronoi formalism has some nice features:  water-tightness is guaranteed for any distribution of site locations, and the facets are always planar and convex.  And, there is the potential for capturing sharp features in a natural way.  For these reasons, we employ Voronoi as a step in our b-rep creation algorithm.

\textcolor{purple}{
Notes from Omar:\\
- When discussing marching cubes/aliasing, we need to acknowledge that our proposed method also exhibits aliasing. But as will be demonstrated in the performance assessment section, to a much lesser degree\\
- Addressing the reviewer comment about our method not being able to handle sparse point clouds - maybe a quick comment somewhere along the lines of: The intention of our method is to produce a dense point cloud from fairly low-resolution image data. By definition, our method produces dense point clouds. So the scope of our surface reconstruction step is as such. And addressing sparse point clouds is out of our scope.\\
- Reviewer comment emphasizing for us to mention moving least squares and this Fleishman reference~\cite{fleishman2005}. ``When discussing the point cloud => mesh step, it seems to me that moving least\\
squares approaches should be mentionned. For instance "Robust Moving Least
Squares Fitting With Sharp Features" by Fleishman et al.''\\
- At some point around us discussing the flowchart, we need to explain why we do not perform smoothing for the Poisson approach. The reason is that screened Poisson surface reconstruction already performs smoothing on the indicator function in its algorithm. In fact, the paper touts ``We also observe that the screened Poisson reconstruction introduces less smoothing, providing a reconstruction that is truer to the original data than either the original Poisson or the SSD reconstructions''. The counter would be that the decimation step introduces error. However, we checked this and decimation does not significantly impact our error measures in the Performance Assessment section. All that said, we did try adding Taubin smoothing to the Poisson results as a last step, and the results are still similar to the proposed method, with the proposed method still performing slightly better for the sphere.
- several of the reviewer comments about other work probably belong here in the introduction rather than in the Limitations section
- both reviewers seemed to think we ``discarded'' Poisson surface reconstruction after mentioning it. Let's be sure not to sound dismissive about it or other methods}